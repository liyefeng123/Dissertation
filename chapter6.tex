\chapter{Conclusion}
Whenever you wish to refer to books or articles relevant to your report
you should use a citation such as \cite{lamport}. You can also force
entries to appear in the bibliography without  a citation appearing in
the document, by using \verb=\nocite=.  

%% This nocite produces 2 entries in the bibliography
\nocite{boyle,rd-only} 

Each document cited must have an entry in a \textsf{.bib} file. For this
document we have only one, called \textsf{refs.bib}. These files are
listed in the \verb=\bibliography= command at the end of
\textsf{report.tex}. Note that the \textsf{.bib} files can (and often
do) contain many more entries than are actually cited in a partcular
document; the only ones that appear in the bibliography are those that
have been referenced using \verb=\cite= or \verb=\nocite=.

In order to generate the appropriate reference entries, you will need
to run \texttt{bibtex} after \texttt{latex} has been run, using the
command \texttt{bibtex report}. This will generate a file
\textsf{report.bbl}, which contains the bibliography entries. Once
that file is there, you do not need to run \texttt{bibtex} again
unless you add new citations, but you will probably have to run
\texttt{latex} twice after running \texttt{bibtex} the first time.

The \TeX{} FAQ (\cite{url-cite}) gives tips on how to cite URLs.

The file \textsf{refs.bib} provides an example of what can be done
with Bib\TeX. You can find much more information in any book on
\LaTeX, for example 

% Local Variables: 
% mode: latex
% TeX-master: "report"
% End: 
